\documentclass[a4paper, 10pt]{article}
\usepackage[a4paper,left=3cm,right=2cm,top=2.5cm,bottom=2.5cm]{geometry}
\usepackage[utf8]{inputenc} % Change according your file encoding
\usepackage{graphicx}
%\usepackage[demo]{graphicx}
\usepackage{url}

\usepackage{float}
\usepackage{amsmath}
\usepackage{xcolor}
\usepackage{todonotes}

\usepackage{listings}

\definecolor{backcolour}{rgb}{0.95,0.95,0.92}

\lstdefinestyle{mystyle}{
    backgroundcolor=\color{backcolour},  
    breakatwhitespace=false,         
    basicstyle=\scriptsize,
    breaklines=true,                 
    captionpos=b,                    
    keepspaces=true,                 
    showspaces=false,                
    showstringspaces=false,
    showtabs=false,                  
    tabsize=2,
    frame=single
}



\lstset{style=mystyle}

%opening
\title{Seminar Report: Opty}
\author{\textbf{Ignacio Encinas Rubio, Adrián Jimenez González}}
\date{\normalsize\today{}}

\begin{document}

\maketitle

%\begin{center}
  %Upload your report in PDF format.
  
  %Use this LaTeX template to format the report.
  
	%A compressed file (.tar.gz) containing all your source code files must be submitted together with this report.
%\end{center}

\section{Introduction}


\textit{Introduce in a couple of sentences the seminar and the main topic related to distributed systems it covers.}

\section{Code modifications}

   In this section we will briefly comment the code added to the template version in order to
   make the algorithm work. We will show the minimum number of lines of code possible to follow the reasoning.

  \subsection{Entry.erl}

    \begin{minipage}{.45\textwidth}
	\begin{lstlisting}[language=erlang, caption={Template}]
entry(Value, Time) ->
    receive
        {read, Ref, From} ->
            %% TODO: ADD SOME CODE
            entry(Value, Time);
        {write, New} ->
            entry(... , make_ref()); 
        {check, Ref, Readtime, From} ->
            if 
                 ... == ... ->  
                    %% TODO: ADD SOME CODE
                true ->
                    From ! {Ref, abort}
            end,
            entry(Value, Time);
        stop ->
            ok
    end.
 	\end{lstlisting}
    \end{minipage}\hfill
    \begin{minipage}{.45\textwidth}
	\begin{lstlisting}[language=erlang, caption={Filled version}]
entry(Value, Time) ->
    receive
        {read, Ref, From} ->
	    From ! {Ref, self(), Value, Time},
            entry(Value, Time);
        {write, New} ->
            entry(New , make_ref());
        {check, Ref, Readtime, From} ->
            if 
                 Readtime == Time -> 
		    From ! {Ref, ok};
                true ->
                    From ! {Ref, abort}
            end,
            entry(Value, Time);
        stop ->
            ok
    end.
	\end{lstlisting}
    \end{minipage}

In this module we need to complete the \textbf{entry} function. This function is the responsible of the behaviour of the entries. The fuction will receive 3 types of message: read, write and check. 

\begin{itemize}
  \item  In case it receive a read message we will send a message with \textit{Value}, a timestamp and our PID.
  \item In case we receive a write message we will just update the \textit{Value}. 
  \item Finnaly, in case we receive a check messsage, we must compare \textit{Readtime} of the message with our timestamp, if it is the same, we send a ok message (commit), else we send an abort message.


\end{itemize}

\clearpage
\section{Performance}

We have used a bash script to run simulations as extensive as possible. It is worth noting that results might contain hidden effects resultant of the specific hardware conditions among other things, and thats why we believe solid conclusions cannot be reached in many cases.

\subsection{Number of concurrenct clients in the system}
\label{sec:numclients}

\subsection{Number of entries in the store}
\label{sec:numentries}

\subsection{Number of reads per transaction}
\label{sec:numreads}

\subsection{Number of writes per transaction}
\label{sec:numwrites}

\subsection{Ratio of read/writes}
\label{sec:ratioreadwrites}

\subsection{Ratio of read/writes}
\label{sec:ratioreadwrites}

\subsection{Different percentage of accessed entries w.r.t the total number of entries}
\label{sec:nose}
no entiendo


\subsection{What is the impact of each of these parameters on the success rate?}

\subsection{Is the success rate the same for the different clients?}

\clearpage
\section{Distributed execution}

\clearpage
\section{Other concurrency control techniques}

\clearpage
\section{Personal opinion}

\subsection{Ignacio}

\subsection{Adrián}

\end{document}
