\documentclass[a4paper, 10pt]{article}
\usepackage[a4paper,left=3cm,right=2cm,top=2.5cm,bottom=2.5cm]{geometry}
\usepackage[utf8]{inputenc} % Change according your file encoding
\usepackage{graphicx}
%\usepackage[demo]{graphicx}
\usepackage{url}

\usepackage{float}
\usepackage{amsmath}
\usepackage{xcolor}
\usepackage{todonotes}

\usepackage{listings}

\definecolor{backcolour}{rgb}{0.95,0.95,0.92}

\lstdefinestyle{mystyle}{
    backgroundcolor=\color{backcolour},  
    breakatwhitespace=false,         
    basicstyle=\scriptsize,
    breaklines=true,                 
    captionpos=b,                    
    keepspaces=true,                 
    showspaces=false,                
    showstringspaces=false,
    showtabs=false,                  
    tabsize=2,
    frame=single
}



\lstset{style=mystyle}

%opening
\title{Seminar Report: Opty}
\author{\textbf{Ignacio Encinas Rubio, Adrián Jimenez González}}
\date{\normalsize\today{}}

\begin{document}

\maketitle

%\begin{center}
  %Upload your report in PDF format.
  
  %Use this LaTeX template to format the report.
  
	%A compressed file (.tar.gz) containing all your source code files must be submitted together with this report.
%\end{center}

\section{Introduction}


\textit{Introduce in a couple of sentences the seminar and the main topic related to distributed systems it covers.}

\section{Code modifications}

   In this section we will briefly comment the code added to the template version in order to
   make the algorithm work. We will show the minimum number of lines of code possible to follow the reasoning.

  \subsection{Entry.erl}

    \begin{minipage}{.45\textwidth}
	\begin{lstlisting}[language=erlang, caption={Template}]
entry(Value, Time) ->
    receive
        {read, Ref, From} ->
            %% TODO: ADD SOME CODE
            entry(Value, Time);
        {write, New} ->
            entry(... , make_ref()); 
        {check, Ref, Readtime, From} ->
            if 
                 ... == ... ->  
                    %% TODO: ADD SOME CODE
                true ->
                    From ! {Ref, abort}
            end,
            entry(Value, Time);
        stop ->
            ok
    end.
 	\end{lstlisting}
    \end{minipage}\hfill
    \begin{minipage}{.45\textwidth}
	\begin{lstlisting}[language=erlang, caption={Filled version}]
entry(Value, Time) ->
    receive
        {read, Ref, From} ->
	    From ! {Ref, self(), Value, Time},
            entry(Value, Time);
        {write, New} ->
            entry(New , make_ref());
        {check, Ref, Readtime, From} ->
            if 
                 Readtime == Time -> 
		    From ! {Ref, ok};
                true ->
                    From ! {Ref, abort}
            end,
            entry(Value, Time);
        stop ->
            ok
    end.
  	\end{lstlisting}
  \end{minipage}

In this module we need to complete the \textbf{entry} function. This function is the responsible of the behaviour of the entries. The fuction will receive 3 types of message: read, write and check. 

\begin{itemize}
  \item  In case it receive a read message we will send a message with \textit{Value}, a timestamp and our PID.
  \item In case we receive a write message we will just update the \textit{Value}. 
  \item Finnaly, in case we receive a check messsage, we must compare \textit{Readtime} of the message with our timestamp, if it is the same, we send a ok message (commit), else we send an abort message.
\end{itemize}


\subsection{Handler.erl}

  \begin{minipage}{.45\textwidth}
	\begin{lstlisting}[language=erlang, caption={Template}]
handler(Client, Validator, Store, Reads, Writes) ->         
    receive
        {read, Ref, N} ->
            case lists:keyfind(..., ..., ...) of  %% TODO: COMPLETE
                {N, _, Value} ->
                    %% TODO: ADD SOME CODE
                    handler(Client, Validator, Store, Reads, Writes);
                false ->
                    %% TODO: ADD SOME CODE
                    %% TODO: ADD SOME CODE
                    handler(Client, Validator, Store, Reads, Writes)
            end;
        {Ref, Entry, Value, Time} ->
            %% TODO: ADD SOME CODE HERE AND COMPLETE NEXT LINE
            handler(Client, Validator, Store, [...|Reads], Writes);
        {write, N, Value} ->
            %% TODO: ADD SOME CODE HERE AND COMPLETE NEXT LINE
            Added = lists:keystore(N, 1, ..., {N, ..., ...}),
            handler(Client, Validator, Store, Reads, Added);
        {commit, Ref} ->
            %% TODO: ADD SOME CODE
        abort ->
            ok
    end.
 	\end{lstlisting}
    \end{minipage}\hfill
    \begin{minipage}{.45\textwidth}
	\begin{lstlisting}[language=erlang, caption={Filled version}]
handler(Client, Validator, Store, Reads, Writes) ->         
    receive
        {read, Ref, N} ->
            case lists:keyfind(N, 1, Writes) of  
                {N, _, Value} ->
		    Client ! {value, Ref, Value},
                    handler(Client, Validator, Store, Reads, Writes);
                false ->
		    Entry = store:lookup(N, Store),
		    Entry ! {read, Ref, self()},
                    handler(Client, Validator, Store, Reads, Writes)
            end;
        {Ref, Entry, Value, Time} ->
	    Client ! {value, Ref, Value},
            handler(Client, Validator, Store, [{Entry, Time}|Reads], Writes);
        {write, N, Value} ->
	    Entry = store:lookup(N, Store),
            NewWrites = lists:keystore(N, 1, Writes, {N, Entry, Value}),
            handler(Client, Validator, Store, Reads, NewWrites);
        {commit, Ref} ->
	    Validator ! {validate, Ref, Reads, Writes, Client};
        abort ->
            ok
    end.
  	\end{lstlisting}
  \end{minipage}

The next function we need to fil is on Handler module. The \textbf{handler} function expects to receive 4 types of messages: read, reply of entry, write and commit message.

\begin{itemize}
  \item If the message is read, we need to see if we have already written to this entry before using N, 1 and Writes as parameters for keyfind function. In case that is true, we will send \textit{Value} to the Client. Else, we look for the corresponding N Entry and we send a read message to the handler.
  \item In case we recieve a reply of an Entry, we need to seend teh reading value of the Entry to the Client. Also we need to add this Read to the Client's Reads.
  \item If the message is write, we look up for an entry and we need to append or update if existed the entry in our Writes list.
  \item Last message we can receive is commit. With that message we need to send a validate message to the Validator with with \textit{Ref, Reads, Writes and Client} values.
\end{itemize}

\clearpage
\subsection{Validator.erl}

In the validator module we have 3 functions to complete. 

\begin{minipage}{.45\textwidth}
	\begin{lstlisting}[language=erlang, caption={Template}]

validator() ->
    receive
        {validate, Ref, Reads, Writes, Client} ->
            Tag = make_ref(),
            send_read_checks(..., Tag),  %% TODO: COMPLETE
            case check_reads(..., Tag) of  %% TODO: COMPLETE
                ok ->
                    update(...),  %% TODO: COMPLETE
                    Client ! {Ref, ok};
                abort ->
                    %% TODO: ADD SOME CODE
            end,
            validator();
        stop ->
            ok;
        _Old ->
            validator()
    end.
 	\end{lstlisting}
    \end{minipage}\hfill
    \begin{minipage}{.45\textwidth}
	\begin{lstlisting}[language=erlang, caption={Filled version}]

validator() ->
    receive
        {validate, Ref, Reads, Writes, Client} ->
            Tag = make_ref(),
            send_read_checks(Reads, Tag),  
            case check_reads(length(Reads), Tag) of  
                ok ->
                    update(Writes),
                    Client ! {Ref, ok};
                abort ->
		    Client ! {Ref, abort}
            end,
            validator();
        stop ->
            ok;
        _Old ->
            validator()
    end.
  	\end{lstlisting}
  \end{minipage}


The \textbf{validator} function we wait for a validate message. When we receive that message we need to check the request to every entry we have read and check the results.

\begin{itemize}
  \item In case the case the transaction is OK, we flush writes and finish with a ok message to the Client.
  \item In case we read stale data, we send to the Client an abort message.
\end{itemize}

\begin{minipage}{.45\textwidth}
	\begin{lstlisting}[language=erlang, caption={Template}]

update(Writes) ->
    lists:foreach(fun({_, Entry, Value}) -> 
                  %% TODO: ADD SOME CODE
                  end, 
                  Writes).
 	\end{lstlisting}
    \end{minipage}\hfill
    \begin{minipage}{.45\textwidth}
	\begin{lstlisting}[language=erlang, caption={Filled version}]

update(Writes) ->
    lists:foreach(fun({_, Entry, Value}) -> 
		  % Update each entry in our Writes
		  Entry ! {write, Value}
                  end, 
                  Writes).
  	\end{lstlisting}
  \end{minipage}


  The next fuction to complete is \textbf{update}. In this fuction we need to send a write message to Entry with the corresponding Value.


\begin{minipage}{.45\textwidth}
	\begin{lstlisting}[language=erlang, caption={Template}]
send_read_checks(Reads, Tag) ->
    Self = self(),
    lists:foreach(fun({Entry, Time}) -> 
                  %% TODO: ADD SOME CODE
                  end, 
                  Reads).
 	\end{lstlisting}
    \end{minipage}\hfill
    \begin{minipage}{.45\textwidth}
	\begin{lstlisting}[language=erlang, caption={Filled version}]

send_read_checks(Reads, Tag) ->
    Self = self(),
    lists:foreach(fun({Entry, Time}) -> 
		  Entry ! {check, Tag, Time, Self}
                  end, 
                  Reads).
  	\end{lstlisting}
  \end{minipage}

Last function is \textbf{send\_read\_checks}. In this function we send to the Entry a check message with the Tag and Time for each existing Entry.

\clearpage

\subsection{Server.erl}

\begin{minipage}{.45\textwidth}
	\begin{lstlisting}[language=erlang, caption={Template}]
server(Validator, Store) ->
    receive 
        {open, Client} ->
            %% TODO: ADD SOME CODE
            server(Validator, Store);
        stop ->
            Validator ! stop,
            store:stop(Store)
    end.
 	\end{lstlisting}
    \end{minipage}\hfill
    \begin{minipage}{.45\textwidth}
	\begin{lstlisting}[language=erlang, caption={Filled version}]

server(Validator, Store) ->
    receive 
        {open, Client} ->
	    % Send client the Validator and Store
	    Client ! {transaction, Validator, Store},
            server(Validator, Store);
        stop ->
            Validator ! stop,
            store:stop(Store)
    end.
  	\end{lstlisting}
  \end{minipage}

  The last module we have to modify is \textbf{server}. In this case \textit{server} function is waiting for open message. When we receive that message, we send to the Client the Validator and Store for the transaction.

\clearpage
\section{Performance}

We have used a bash script to run simulations as extensive as possible. It is worth noting that results might contain hidden effects resultant of the specific hardware conditions among other things, and thats why we believe solid conclusions cannot be reached in many cases.

\subsection{Number of concurrenct clients in the system}
\label{sec:numclients}

\subsection{Number of entries in the store}
\label{sec:numentries}

\subsection{Number of reads per transaction}
\label{sec:numreads}

\subsection{Number of writes per transaction}
\label{sec:numwrites}

\subsection{Ratio of read/writes}
\label{sec:ratioreadwrites}

\subsection{Ratio of read/writes}
\label{sec:ratioreadwrites}

\subsection{Different percentage of accessed entries w.r.t the total number of entries}
\label{sec:nose}
no entiendo


\subsection{What is the impact of each of these parameters on the success rate?}

\subsection{Is the success rate the same for the different clients?}

\clearpage
\section{Distributed execution}

\clearpage
\section{Other concurrency control techniques}

\clearpage
\section{Personal opinion}

\subsection{Ignacio}

\subsection{Adrián}

\end{document}
