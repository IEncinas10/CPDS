\documentclass[a4paper, 10pt]{article}
\usepackage[a4paper,left=2.5cm,right=2cm,top=2cm,bottom=2cm]{geometry}
\usepackage[utf8]{inputenc} % Change according your file encoding
\usepackage{graphicx}
%\usepackage[demo]{graphicx}
\usepackage{url}

\usepackage{float}
\usepackage{amsmath}
\usepackage{xcolor}
\usepackage{todonotes}

\usepackage{listings}

\definecolor{backcolour}{rgb}{0.95,0.95,0.92}

\lstdefinestyle{mystyle}{
    backgroundcolor=\color{backcolour},  
    breakatwhitespace=false,         
    basicstyle=\scriptsize,
    breaklines=true,                 
    captionpos=b,                    
    keepspaces=true,                 
    showspaces=false,                
    showstringspaces=false,
    showtabs=false,                  
    tabsize=2,
    frame=single,
    morekeywords = {MPI_Comm_size},
}



\lstset{style=mystyle}

%opening
\title{\textbf{Parallelism: Assignment 2\\Solvind the Heat Equation \\using several Parallel Programming Models}}
\author{Ignacio Encinas Rubio, Adrián Jiménez González\\\{ignacio.encinas,adrian.jimenez.g\}.estudiantat.upc.edu}
\date{\normalsize\today{}}

\begin{document}

\maketitle

%\begin{center}
  %Upload your report in PDF format.
  
  %Use this LaTeX template to format the report.
  
	%A compressed file (.tar.gz) containing all your source code files must be submitted together with this report.
%\end{center}

\section{Introduction}

\section{Parallelization}

In this section we will explain the steps we followed to achieve the parallelization of both solvers, Jacobi and Gauss-Seidel. In the following sections, we are going to show how the parallel codes have been done in each Parallel Programming Model for each solver.

\subsection{OpenMP}

  For parallelize the code for Jacobi solver with OpenMP, we just needed to modify the \textit{solver-omp.c} file. The \textit{heat-omp.c} file did not need any modification. Jacobi solver does not have any data dependency, so the parallelization can be achieved with a \texttt{\#pragma} shown above.

\begin{lstlisting}[language=c, caption={OpenMP pragma for Jacobi parallelization}]
    #pragma omp parallel for collapse(2) private(diff) reduction(+:sum)
    for (int ii=0; ii<nbx; ii++)
        for (int jj=0; jj<nby; jj++) 
            for (int i=1+ii*bx; i<=min((ii+1)*bx, sizex-2); i++) 
                for (int j=1+jj*by; j<=min((jj+1)*by, sizey-2); j++) {
	                  utmp[i*sizey+j]= 0.25 * (u[ i*sizey     + (j-1) ]+  // left
					          u[ i*sizey     + (j+1) ]+  // right
				            u[ (i-1)*sizey + j     ]+  // top
				            u[ (i+1)*sizey + j     ]); // bottom
	            diff = utmp[i*sizey+j] - u[i*sizey + j];
	            sum += diff * diff; 
	        }
\end{lstlisting}

  For Jacobi, we do not need the first 2 outer \texttt{for} loops to be done in ``blocks", this will be useful for Gauss-Seidel. So we will fuse both fors with the \texttt{collapse(2)} clause. The next clause we use is \texttt{private(diff)} in which we specify that variable to be allocated for each thread to have their own value per thread. Last, we have the \texttt{reduction(+:sum)} specifying the way we want to reduce the values obtained by each thread. In this case, we want to sum (+) the values for each \texttt{sum} when we finish the parallel section in order to return the residual of the solver.


\subsection{MPI}

\subsection{CUDA}

\section{Parallel solution}

\subsection{OpenMP}

\subsection{MPI}

\subsection{CUDA}


\end{document}
